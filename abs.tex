\documentclass{beamer}
\usepackage{hyperref}
\usepackage{listings}
\usepackage{makecell,multirow}
\usepackage{array}
\usepackage{varwidth}
\usepackage{xcolor}
\usepackage{lmodern}
\usepackage{graphicx}
\usepackage{tikz}
\usepackage[siunitx, RPvoltages]{circuitikz}
\usepackage{pgfplots}

\usetikzlibrary{positioning}

\usepgfplotslibrary{external}
\tikzexternalize


\definecolor{commentsColor}{rgb}{0.497495, 0.497587, 0.497464}
\definecolor{keywordsColor}{rgb}{0.000000, 0.000000, 0.635294}
\definecolor{stringColor}{rgb}{0.558215, 0.000000, 0.135316}

\lstset{ %
  backgroundcolor=\color{white},   % choose the background color; you must add \usepackage{color} or \usepackage{xcolor}
  basicstyle=\footnotesize,        % the size of the fonts that are used for the code
  breakatwhitespace=false,         % sets if automatic breaks should only happen at whitespace
  breaklines=true,                 % sets automatic line breaking
  captionpos=b,                    % sets the caption-position to bottom
  commentstyle=\color{commentsColor}\textit,    % comment style
  deletekeywords={...},            % if you want to delete keywords from the given language
  escapeinside={\%*}{*)},          % if you want to add LaTeX within your code
  extendedchars=true,              % lets you use non-ASCII characters; for 8-bits encodings only, does not work with UTF-8
  frame=tb,                                % adds a frame around the code
  keepspaces=true,                 % keeps spaces in text, useful for keeping indentation of code (possibly needs columns=flexible)
  keywordstyle=\color{keywordsColor}\bfseries,       % keyword style
  language=c++,                 % the language of the code (can be overrided per snippet)
  otherkeywords={*,...},           % if you want to add more keywords to the set
  numbers=left,                    % where to put the line-numbers; possible values are (none, left, right)
  numbersep=5pt,                   % how far the line-numbers are from the code
  numberstyle=\tiny\color{commentsColor}, % the style that is used for the line-numbers
  rulecolor=\color{black},         % if not set, the frame-color may be changed on line-breaks within not-black text (e.g. comments (green here))
  showspaces=false,                % show spaces everywhere adding particular underscores; it overrides 'showstringspaces'
  showstringspaces=false,          % underline spaces within strings only
  showtabs=false,                  % show tabs within strings adding particular underscores
  stepnumber=1,                    % the step between two line-numbers. If it's 1, each line will be numbered
  stringstyle=\color{stringColor}, % string literal style
  tabsize=2,                       % sets default tabsize to 2 spaces
  title=\lstname,                  % show the filename of files included with \lstinputlisting; also try caption instead of title
  columns=fixed                    % Using fixed column width (for e.g. nice alignment)
}

\def\labelsize{\fontsize{7pt}{8pt}\selectfont}

\title{An ECU Design for an Anti-Lock Braking System}
\author{Xu, Daniel}
\institute{Revature}
\date{October 6, 2023}

\begin{document}
\frame{\titlepage}

\begin{frame}
  \frametitle{Intended End Result}
  The design of an anti-lock braking system (ABS) should prevent the wheels
  from locking up while braking, allowing the driver to retain tractive
  contract with the road, as well as control of the car.
\end{frame}

\begin{frame}
  \frametitle{Skidding}
  \begin{itemize}
  \item The most common reason why vehicles skid is due to tires
    receiving more force than they're meant to take, leading to the surface of
    tire sliding across the road
  \item The most force is usually applied to tires when braking, accelerating, or
    cornering
  \end{itemize}
\end{frame}


\begin{frame}
  \frametitle{How our ECU achieves its function}
  \begin{enumerate}
  \item
    Our ECU monitors the speed sensors constantly. It is looking for
    sudden decelerations in the wheel that are not normal. Right before
    a wheel locks up, it experiences a rapid deceleration. If not
    checked, the wheel would suddenly stop faster than any car could.

  \item The ABS ECU decreases pressure to the brake via a valve until it
    sees an acceleration, and then raises the pressure until it sees a
    deceleration again. The ECU is capable of doing this before the wheel
    experiences a significant change in speed.

%   \item Manually pumping the brakes is therefore not needed when driving
%     on a slippery or low friction surface, allowing the driver to steer
%     even in the most dire of braking conditions
%
%   \item While the ABS is active, the driver will feel a pulsing in the brake
%     pedal; this is due to the sudden opening and closing of the valves, and
%     notifies the driver that the ABS is in operation
%
  \end{enumerate}
\end{frame}

\begin{frame}
  \frametitle{Components of an ABS}
  \begin{itemize}
  \item \textbf{Wheel Speed Sensor} - The faster a wheel is spinning, the higher
    the voltage of a wheel speed sensor; this info is sent to the ECU so that
    the speed and then acceleration can be detected
  \item \textbf{Valves} - Helps regulate the pressure in a brake, giving it a degree
    of control over the acceleration. Some valves have three positions:
    \begin{itemize}
    \item \textbf{Open} - pressure from master cyclinder is can flow to brake
    \item \textbf{Blocking} - prevents pressure from master cylinder to getting to brakes
    \item \textbf{Released} - releases pressure from the brake
    \end{itemize}
  \item \textbf{Pump} - Returns pressure to brakes after the valves have released some pressure
  \item \textbf{ECU} - Microcontroller responsible for coordinating the entire ABS
  \end{itemize}
\end{frame}

\begin{frame}
  \frametitle{Converting Sensor Data into Real Information}

  Let \(V\) be voltage and \(s\) be speed. According to \cite{ws24}, the Bendix
  WS-24 antilock wheel lock speed sensor produces a voltage of 0.350 VAC
  at a minimum of 100 Hz (about 7 mph) at a gap of 0.28 inches. So for every
  7 mph increase in speed, the wheel speed sensor should produce 0.350 higher
  volts. 7 mph is 3.12928 m/s, so for a given voltage produced by the speed sensor,
  the speed of a wheel can be calculated as so:

  \[\frac{8.9408 \, \textrm{m / s}}{1 \, \textrm{VAC}} \cdot V = s \]
\end{frame}

\begin{frame}
  \frametitle{Wheel Speed Sensor}
  \begin{tikzpicture}[domain=0:6]
    \begin{axis}[
        title = {Voltage vs Vehicle Speed},
        axis lines = left,
        xlabel = {Voltage (VAC)},
        ylabel = {Speed (m/s)}
      ]
      \addplot[
        color=red,
        domain=0:6,
        samples=100
      ]{8.9408 * x};
    \end{axis}

  \end{tikzpicture}
\end{frame}

\begin{frame}
  \frametitle{Circuit Diagram}
  \begin{circuitikz}[american]
    \draw (8, 0) node[dipchip, hide numbers, font=\scriptsize, num pins=16, external pins width=0](E){ECM};
    \node [left, font=\tiny] at (E.bpin 16) {S0};
    \node [left, font=\tiny] at (E.bpin 15) {S1};
    \node [left, font=\tiny] at (E.bpin 14) {S2};
    \node [left, font=\tiny] at (E.bpin 13) {S3};

    \node [left, font=\tiny] at (E.bpin 12) {M0};
    \node [left, font=\tiny] at (E.bpin 11) {M1};
    \node [left, font=\tiny] at (E.bpin 10) {M2};
    \node [left, font=\tiny] at (E.bpin 9) {M3};

    \node[rectangle, font=\scriptsize, align=center, draw=black] (dash) at (5, -1.96) { Dashboard };

    \node[rectangle, font=\tiny, align=center, text width=0.8cm, draw=black] (fls) at (10, 3) {
      Front Left
      Sensor
    };

    \node[rectangle, font=\tiny, align=center, text width=0.8cm, draw=black] (frs) at (11.3, 3) {
      Front Right
      Sensor
    };

    \node[rectangle, font=\tiny, align=center, text width=0.8cm, draw=black] (bls) at (12.6, 3) {
      \labelsize
      Back Left
      Sensor
    };

    \node[rectangle, font=\tiny, align=center, text width=0.8cm, draw=black] (brs) at (13.9, 3) {
      \labelsize
      Back Right
      Sensor
    };

    \node[rectangle, font=\tiny, align=center, text width=0.8cm, draw=black] (flm) at (10, -3) {
      Front Left
      Modulator Valve
    };

    \node[rectangle, font=\tiny, align=center, text width=0.8cm, draw=black] (frm) at (11.3, -3) {
      Front Right
      Modulator Valve
    };

    \node[rectangle, font=\tiny, align=center, text width=0.8cm, draw=black] (blm) at (12.6, -3) {
      \labelsize
      Back Left
      Modulator Valve
    };

    \node[rectangle, font=\tiny, align=center, text width=0.8cm, draw=black] (brm) at (13.9, -3) {
      \labelsize
      Back Right
      Modulator Valve
    };

    \draw (fls.north) -- ++ (0, 0.02) node[ground,rotate=180]{};
    \draw (frs.north) -- ++ (0, 0.02) node[ground,rotate=180]{};
    \draw (bls.north) -- ++ (0, 0.02) node[ground,rotate=180]{};
    \draw (brs.north) -- ++ (0, 0.02) node[ground,rotate=180]{};

    \draw (flm.south) -- ++ (0, -0.02) node[ground]{};
    \draw (frm.south) -- ++ (0, -0.02) node[ground]{};
    \draw (blm.south) -- ++ (0, -0.02) node[ground]{};
    \draw (brm.south) -- ++ (0, -0.02) node[ground]{};


    \draw (E.bpin 1) -- ++ (-0.8, 0) to[normal open switch, mirror, n=Swp, font=\tiny, l_=Ignition Switch] ++
    (-0.8, 0) to[battery, invert, font=\tiny, l_=12V] ++ (-2, 0) node[ground]{};

    \draw (E.bpin 16) -- ++ (1.16, 0) -- (fls.south);
    \draw (E.bpin 15) -- ++ (2.46, 0) -- (frs.south);
    \draw (E.bpin 14) -- ++ (3.76, 0) -- (bls.south);
    \draw (E.bpin 13) -- ++ (5.06, 0) -- (brs.south);

    \draw (E.bpin 12) -- ++ (1.16, 0) -- (flm.north);
    \draw (E.bpin 11) -- ++ (2.46, 0) -- (frm.north);
    \draw (E.bpin 10) -- ++ (3.76, 0) -- (blm.north);
    \draw (E.bpin 9) -- ++ (5.06, 0) -- (brm.north);

    \draw (E.bpin 8) -- (dash.east);

    \node[font=\tiny] at (6.5, 1.6) { KL15 };

  \end{circuitikz}
\end{frame}

\begin{frame}
  \frametitle{Potential DTC Codes involving an ABS}
  \begin{description}
    \item[U0121] Lost Communication with ABS Control Module
    \item[P0863] TCM Communication Circuit
    \item[P0883] TCM Power Input Signal High
    \item[P0864] TCM Communication Circuit Range/Performance
    \item[P215A] Vehicle Speed - Wheel Speed Correlation
  \end{description}

\end{frame}
\begin{frame}[fragile]
  \frametitle{Pseudocode (1 of 4)}
  \begin{lstlisting}[language=c++, basicstyle=\tiny, columns=fullflexible]
    enum abs_status { ABS_NORMAL, ABS_DECEL, ABS_ACCEL };

    typedef enum abs_status AbsStatus;

    // speed should be in m/s and time should be in s
    struct wheel_sensor {
      int pin;
      double prev_speed;
      double prev_time;
      double curr_speed;
      double curr_time;
      AbsStatus status;
    };

    typedef struct wheel_sensor WheelSensor;

    void ws_init(WheelSensor *ws, int pin) {
      ws->pin = pin;
      ws->prev_speed = 0.0, ws->curr_speed = 0.0;
      ws->prev_time = 0.0, ws->curr_time = 0.0;
      ws->status = ABS_NORMAL;
    }

    \end{lstlisting}
\end{frame}

\begin{frame}[fragile]
  \frametitle{Pseudocode (2 of 4)}
  \begin{lstlisting}[language=c++, basicstyle=\tiny, columns=fullflexible]
    #define VOLTS_TO_METERS_PERS_SECOND_CONV_CONST 8.9408

    static inline double ws_voltage_to_meters_per_second(double volts) {
      return VOLTS_TO_METERS_PERS_SECOND_CONV_CONST  * volts;
    }

    static inline double millis_to_seconds(double mil) {
      return mil / 1000.0;
    }

    void ws_status_update(WheelSensor *ws) {
      double accel = ws_acceleration(ws);

      if (ws->status == ABS_NORMAL) {
        if (accel < WS_DECEL_THRESHOLD) {
          ws->status = WS_DECEL;
        }
      } else if (ws->status == ABS_DECEL) {
        if (accel > 0.0) {
          ws->status = WS_ACCEL;
        }
      } else { // currently accelerating, want to decelerate until normal
        if (accel < 0.0) {
          ws->status = WS_NORMAL;
        }
      }
    }

  \end{lstlisting}
\end{frame}

\begin{frame}[fragile]
  \frametitle{Pseudocode (3 of 4)}
  \begin{lstlisting}[language=c++, basicstyle=\tiny, columns=fullflexible]
    void ws_update(WheelSensor *ws) {
      ws->prev_speed = ws->curr_speed;
      ws->prev_time = ws->curr_time;

      ws->curr_speed = ws_voltage_to_meters_per_second(AnalogRead(ws->pin));
      ws->curr_time = millis_to_seconds((double) millis());

      ws_status_update(ws);
    }


    inline double ws_acceleration(WheelSensor *ws) {
      return (ws->curr_speed - ws->prev_speed) / (ws->curr_time - ws->prev_time);
    }

    WheelSensor fl_ws, fr_ws, bl_ws, br_ws;

    void ABS_setup() {
      ws_init(&fl_ws, FRONT_LEFT_WS_PIN);
      ws_init(&fr_ws, FRONT_RIGHT_WS_PIN);
      ws_init(&bl_ws, BACK_LEFT_WS_PIN);
      ws_init(br_ws, BACK_RIGHT_WS_PIN);
    }

\end{lstlisting}
\end{frame}

\begin{frame}[fragile]
  \frametitle{Pseudocode (4 of 4)}
    \begin{lstlisting}[language=c++, basicstyle=\tiny, columns=fullflexible]

    void ABS_loop() {
      if (fl_ws.status == ABS_DECEL) {
        valve_set_position(FRONT_LEFT_VAVLE_PIN, VALVE_RELEASE);
      } else if (fl_ws.status == ABS_ACCEL) {
        valve_set_position(FRONT_LEFT_VALVE_PIN, VALVE_OPEN);
      }

      if (fl_ws.status == ABS_DECEL) {
        valve_set_position(FRONT_RIGHT_VALVE_PIN, VALVE_RELEASE);
      } else if (fl_ws.status == ABS_ACCEL) {
        valve_set_position(FRONT_RIGHT_VALVE_PIN, VALVE_OPEN);
      }

      if (fl_ws.status == ABS_DECEL) {
        valve_set_position(BACK_LEFT_VALVE_PIN, VALVE_RELEASE);
      } else if (fl_ws.status == ABS_ACCEL) {
        valve_set_position(BACK_LEFT_VALVE_PIN, VALVE_OPEN);
      }

      if (fl_ws.status == ABS_DECEL) {
        valve_set_position(BACK_RIGHT_VALVE_PIN, VALVE_RELEASE);
      } else if (fl_ws.status == ABS_ACCEL) {
        valve_set_position(BACK_RIGHT_VALVE_PIN, VALVE_OPEN);
      }

      ws_update(&fl_ws);
      ws_update(&fr_ws);
      ws_update(&bl_ws);
      ws_update(&br_ws);
    }
  \end{lstlisting}
\end{frame}

\begin{frame}
  \frametitle{References}
  \begin{thebibliography}{9}
    \setbeamertemplate{bibliography item}[online]
  \bibitem{howstuffworks}
    Nice, Karim. ``How Anti-Lock Brakes Work.'' HowStuffWorks, August 23, 2000.
    \url{https://auto.howstuffworks.com/auto-parts/brakes/brake-types/anti-lock-brake.htm}.
    \setbeamertemplate{bibliography item}[online]
    \bibitem{ws24}
    ``Bendix WS-24 Antilock Wheel Speed Sensor.'' Bendix, April 2019.
      \url{https://static.nhtsa.gov/odi/tsbs/2019/MC-10163232-9999.pdf}.

    \bibitem{ec80}
      ``Bendix EC-80 ABS/ATC Electronic Controllers.'' Bendix, April 2019.
      \url{https://static.nhtsa.gov/odi/tsbs/2022/MC-10208811-0001.pdf}
  \end{thebibliography}
\end{frame}

\end{document}
